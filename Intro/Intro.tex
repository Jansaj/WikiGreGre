\begin{abstract}

Le terme grenouille est un nom vernaculaire attribué à certains amphibiens, principalement dans le genre Rana. 
À un de ses stades de développement, la larve de la grenouille est appelée un têtard. 
Les grenouilles sont des quadrupèdes de l'ordre des anoures, tout comme les rainettes, qui sont en général plus vertes et arboricoles, les crapauds dont la peau est plus granuleuse et les xénopes strictement aquatiques. 
Tous ces termes usuels correspondent à des apparences extérieures plus qu'à des classements strictement taxinomiques.

En Europe, parmi les espèces de grenouilles les plus connues figurent la Grenouille verte et la Petite grenouille verte, la Grenouille des champs, la Grenouille rousse et, en élevage, la Grenouille rieuse.

Certaines espèces comme la Grenouille-taureau d'Amérique du Nord, la Grenouille Goliath d'Afrique ou Litoria infrafrenata (grenouille géante) sont remarquables pour leur très grande taille.

Il existe environ 3 800 espèces de grenouilles et crapauds1 qui subissent depuis le milieu du xxe siècle un déclin brutal, déroutant et alarmant.

Plusieurs espèces de grenouilles sont élevées pour consommer la chair de leurs cuisses, d'autres servent à l'expérimentation, d'autres encore sont parfois adoptées pour l'agrément.

Elles sont souvent évoquées dans les textes anciens et présentes dans les représentations artistiques. 
La grenouille est aussi un personnage important du folklore populaire ou enfantin sous forme d'animal tantôt répugnant et maléfique ou, au contraire, magique et bénéfique, en particulier à travers le mythe du prince ou de la princesse transformés en grenouille (ou le plus souvent en crapaud).

\end{abstract}