\section{Étymologie et nomenclature}

La racine du mot $\ll$ grenouille $\gg$ vient du latin rana, voulant dire grenouille, et ranucula ou ranunculus, petite grenouille. Utilisé dès l'époque médiévale sous sa forme ancienne $\ll$ renoille $\gg$ ou $\ll$ grenoille $\gg$ au XIIIe siècle, le mot $\ll$ grenouille $\gg$ est attesté à partir du début du XVIe siècle. Le $\ll$ g $\gg$ initial ayant sans doute été ajouté par évocation du cri guttural de ces animaux2.

Le mot $\ll$ grenouille $\gg$ est déjà présent dans les dictionnaires de français anciens en 1606. Dès sa première édition, en 1694, le Dictionnaire de L'Académie française en donne une définition surprenante : $\ll$ Insecte (sic) qui vit ordinairement dans les marais $\gg$. Insecte est corrigé en $\ll$ petit animal $\gg$ dans la quatrième édition de 1762 avec comme précision $\ll$ quadrupède et ovipare $\gg$ dans sa sixième édition. Il faut attendre la huitième édition de 1932 pour que la grenouille soit mentionnée comme appartenant à $\ll$ l'ordre des Batraciens $\gg$ (désormais ordre des amphibiens)3.

Diderot et d'Alembert, dans l' Encyclopédie ou Dictionnaire raisonné des sciences, des arts et des métiers (1751 à 1772) décrivent d'abord la grenouille comme un $\ll$ animal qui a quatre piés, qui respire par des poumons, qui n'a qu'un ventricule dans le cœur, \& qui est ovipare $\gg$, en distinguant les grenouilles aquatiques des rainettes arboricoles4.

La grenouille coasse. Il ne faut pas confondre avec le cri du corbeau qui croasse.

Une grenouillette est une petite grenouille5.

La larve de la grenouille s'appelle un têtard.

Parmi les amphibiens, on distinguait autrefois spontanément les crapauds des grenouilles, nom donné à d'autres espèces d'anoures, les premiers étaient caractérisés par une peau plus rugueuse, voire pustuleuse, un œil à pupille horizontale, un museau arrondi6, des pattes plus courtes, une moindre capacité à sauter, une marche plus lente, et le fait qu'ils passent moins de temps dans le milieu aquatique que les grenouilles7. Toutes les langues n'utilisent pas des dénominations particulières pour désigner les espèces d'anoures appelées en français sonneur, grenouille, rainette et crapaud. Certaines langues peuvent faire une distinction analogues comme l'anglais avec toad et frog, mais il n'y a pas forcément de correspondance pour une espèce, autrement dit il est abusif de traduire systématiquement frog par grenouille. 